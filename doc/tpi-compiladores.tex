%++++++++++++++++++++++++++++++++++++++++
% Don't modify this section unless you know what you're doing!
\documentclass[letterpaper,12pt]{article}
\usepackage{tabularx} % extra features for tabular environment
\usepackage{amsmath}  % improve math presentation
\usepackage{graphicx} % takes care of graphic including machinery
\usepackage[margin=1in,letterpaper]{geometry} % decreases margins
\usepackage{cite} % takes care of citations
\usepackage[portuguese]{babel}
\usepackage[final]{hyperref} % adds hyper links inside the generated pdf file
\hypersetup{
	colorlinks=true,       % false: boxed links; true: colored links
	linkcolor=blue,        % color of internal links
	citecolor=blue,        % color of links to bibliography
	filecolor=magenta,     % color of file links
	urlcolor=blue         
}
%++++++++++++++++++++++++++++++++++++++++


\begin{document}

\title{%
  Constru\c{c}\~ao de Compiladores I\\
  \large Atividade: Mini-Linguagem}
\author{Bernardo Marotta e Melina Lopes}
\date{\today}
\maketitle

\begin{abstract}
A proposta dessa atividade \'e especificar uma mini linguagem de programa\c{c}\~ao simples que aborde alguns dos conceitos apresentados em sala de aula. 
\end{abstract}


\section{Sum\'ario}
\textbf{A linguagem}\\
\textbf{Aspectos l\'exicos}\\
\textbf{Aspectos sint\'aticos}\\
\textbf{Aspectos sem\^anticos}\\
\textbf{Biblioteca padr\~ao}\\
\textbf{Exemplos}\\


\section{A linguagem}
A linguagem \textbf{PANDA} (Precise Algorithm and Data Architecture) \'e uma mini-linguagem constru\'ida na disciplina de Constru\c{c}\~ao de Compiladores I. \'E uma linguagem imperativa simples e de tipagem est\'atica. Na linguagem, existem tipos primitivos com inteiro e ponto flutuante e tipos compostos, como arrays e estruturas. Existem tamb\'em operadores aritm\'eticos, relacionais, l\'ogicos, espressões etc.

\section{Aspectos l\'exicos}
Coment\'arios de linha: Come\c{c}am com o caracter \# e comentam a linha inteira\\
Coment\'arios de bloco: Abre com ++[ e fecha com ]++\\
Caracteres brancos: N\~ao interferem no interpretador l\'exico (servem de separador)\\
Literais inteiros: Sequ\^encia de d\'igitos decimais\\
Literais booleanos: true e false\\	
Literais string: Delimitado por aspas duplas "", dentro da string, pode ser usado qualquer caracter especial, usando \$ para come\c{c}ar e terminar a sequ\^encia a ser escapada.\\
Identificadores: Sequ\^encia de letras e/ou d\'igitos. Pode conter alguns caracteres especiais como \_, -, \&, etc. 

\section{Aspectos sint\'aticos}
A fim de mostrar os aspectos sint\'aticos, ser\'a apresentada uma GLC que define a sintaxe das constru\c{c}\~oes da linguagem.\\ 
\\
\textbf{Declara\c{c}\~ao de fun\c{c}\~oes:}\\            
Program$\,\to\,$Functions\\
Functions$\,\to\,$Function\\
Functions$\,\to\,$Function Functions\\
Function$\,\to\,$Type (Types): Exp end\\
Type$\,\to\,$bool id\\
Type$\,\to\,$int id\\
Type$\,\to\,$string id\\
Type$\,\to\,$float id\\
Types$\,\to\,$Type TypeRest\\
TypeRest$\,\to\,$ , Type TypeRest\\
TypeRest$\,\to\,$$\lambda$\\
\\
\textbf{Tipos literais:}\\
Exp$\,\to\,$id\\
Exp$\,\to\,$id = Exp\\
Exp$\,\to\,$Lit\\
Exp$\,\to\,$Type[Lit]\\
Lit$\,\to\,$lint\\
Lit$\,\to\,$lstring\\
Lit$\,\to\,$lbool\\
Lit$\,\to\,$lfloat\\
Lit$\,\to\,$$\lambda$\\
Lits$\,\to\,$Lit LitRest\\
LitRest$\,\to\,$ , Lit LitRest\\
LitRest$\,\to\,$$\lambda$\\
\\
\textbf{Opera\c{c}\~oes aritm\'eticas:}\\
Exp$\,\to\,$Exp + Exp\\
Exp$\,\to\,$Exp - Exp\\
Exp$\,\to\,$Exp * Exp\\
Exp$\,\to\,$Exp \% Exp\\
Exp$\,\to\,$Exp / Exp\\
\\
\textbf{Opera\c{c}\~oes relacionais:}\\
Exp$\,\to\,$Exp = Exp\\
Exp$\,\to\,$Exp $>$ Exp\\
Exp$\,\to\,$Exp $<$ Exp\\
Exp$\,\to\,$Exp $<=$ Exp\\
Exp$\,\to\,$Exp $>=$ Exp\\
\\
\textbf{Opera\c{c}\~oes l\'ogicas:}\\
Exp$\,\to\,$Exp and Exp\\
Exp$\,\to\,$Exp or Exp\\
\\
\textbf{Chamada:}\\
Exp$\,\to\,$id (Exp)\\
\\
\textbf{Express\~oes:}\\
Exp$\,\to\,$if Exp : Exp else Exp end\\
Exp$\,\to\,$while Exp : end\\
Exp$\,\to\,$for Exp in Exp : end\\
Exp$\,\to\,$for Exp , Exp : end\\
Exp$\,\to\,$Exps \\
Exps$\,\to\,$Exp ExpRest\\
Exps$\,\to\,$ , Exp ExpRest\\
Exps$\,\to\,$ + Exp ExpRest\\ 
Exps$\,\to\,$$\lambda$\\
\\
\textbf{Preced\^encia:}\\
*,/,\% \\
+, -\\
=, $>$, $<$, $>=$, $<=$ \\
and\\
or\\
else, then, :\\
\section{Aspectos sem\^anticos}
\textbf{Atribui\c{c}\~ao:}\\
A atribui\c{c}\~ao \'e feita atrav\'es do operador =.\\
Ex: a = 5\\
\\
\textbf{Operadores:}\\
O uso de operadores \'e feito de forma simples, com as express\~oes dos dois lados do operador.\\
Ex: a $<$ 5, a $>$ 5, a $>$ 5 and b $<$ 10\\
\\
\textbf{Declara\c{c}\~ao:}\\
A declara\c{c}~ao de vari\'aveis ou fun\c{c}\~oes segue o padr\~ao:\\
Ex: a = 5,\\
a[] = [1,2,3]\\
a[1] = [0]\\
bool function(int)\\
\\
\textbf{Condicionais:}\\
Sendo c1 uma sequ\^encia de comandos e c2 outra:\\
if (BOOL) : c1 end\\
else c2 end\\
\\
\textbf{Repeti\c{c}\~ao:}\\
while (BOOL):\\
    c1\\
end\\
\\
for (ITEM) in (LISTA):\\
    c2\\
end\\
\\
for (i) , (i2):\\
    c3\\
end\\
\\
\section{Biblioteca padr\~ao}
A id das fun\c{c}\~oes da biblioteca padr\~ao come\c{c}am com pnd.\\
Exemplo: pnd.getLine(Exp)\hspace*{8mm}Guarda a entrada do teclado na vari\'avel id (entre par\^enteses).\\
Outros exemplos:\\
pnd.print(Exp)\hspace*{24mm}Imprime na tela o conte\'udo\\
pnd.pow(Exp, Exp)\hspace*{24mm}Calcula pot\^encia\\
pnd.fibonacci(Exp)\hspace*{24mm}Calcula fibonacci\\
pnd.factorial(Exp)\hspace*{24mm}Calcula fatorial\\
\\
\section{Exemplos}
Fun\c{c}\~ao Fibonacci em PANDA:\\
\\
function fibonacci(int n):\\
\hspace*{8mm}a = 0\\
\hspace*{8mm}b = 1\\
\hspace*{8mm}for i = 1, n:\\
\hspace*{12mm}a = b\\
\hspace*{12mm}b = a + b\\
\hspace*{8mm}end\\
\hspace*{8mm}return a\\
end\\
\\
function insertionSort(int array[], int n):\\
\hspace*{8mm}int i, key, j;\\
\hspace*{8mm}for 1, n:\\
\hspace*{12mm}key = arr[i]\\
\hspace*{12mm}j = i-1\\
\hspace*{12mm}while (j >= 0 and arr[j] > key)\\
\hspace*{16mm}arr[j+1] = arr[j]\\
\hspace*{16mm}j = j-1\\
\hspace*{12mm}end\\
\hspace*{12mm}arr[j+1] = key\\
\hspace*{8mm}end\\
end\\
\\
int main():\\
\hspace*{8mm}int a\\
\hspace*{8mm}pnd.getLine(a)\\
\hspace*{8mm}fibonacci(a)\\
\hspace*{8mm}pnd.print("The fibonacci of a is" , a)\\
end\\
\end{document}
